\section{Results}

Using the optimization function, we obtained the following set of parameters:

\begin{align*}
	q_0 &= \begin{bmatrix}
		0.3308 & -0.3497 & 0.0004
	\end{bmatrix} \\
	\dot{q_0} &= \begin{bmatrix}
		0.0003 & 0.0002 & 8.3517
	\end{bmatrix} \\
	k_{p1} &= 450.4380 \\
	k_{p2} &= 101.7156 \\
	k_{d1} &= 94.5593 \\
	k_{d2} &= 4.6025 \\
	\alpha &= 0.1846
\end{align*}

This gives us when computed a higher maximum and minimum velocity for the hip and swing foot. Furthermore, the input for $u_1$ and $u_2$ are also higher, while the angles for the three components of $q$ are smaller. This lead to a lower distance walked by the model over 9 steps, compared to the initial parameters. The nine steps are achieved faster but at the cost of smaller steps. The different metrics can be found here and are to be compared with those of the initial parameters when using the \verb|run\simulation.m| script.

\begin{figure}[h!]
	\centering
	\includegraphics[width=0.8\textwidth]{optimal/angle_velocity_vs_angle}
	\caption{The angle velocity along the angle.}
\end{figure}

The maps appear to be more stable.

\begin{figure}[h!]
	\centering
	\includegraphics[width=0.8\textwidth]{optimal/angle_vs_time.png}
	\caption{The angle of each component of $q$ throughout the simulation.}
\end{figure}

\begin{figure}[h!]
	\centering
	\includegraphics[width=0.99\textwidth]{optimal/displacement_vs_step_number}
	\caption{The displacement of the simulation over each steps.}
\end{figure}

\begin{figure}[h!]
	\centering
	\includegraphics[width=0.99\textwidth]{optimal/speed_vs_time}
	\caption{The angle velocity along the angle.}
\end{figure}

The maps appear to be more stable.

\begin{figure}[h!]
	\centering
	\includegraphics[width=0.99\textwidth]{optimal/torque_vs_time}
	\caption{The controller input over time during the simulation}
\end{figure}